% Basic stuff
\documentclass{article}
\usepackage[utf8]{inputenc}
\usepackage[a4paper, total={6.5in, 9.5in}]{geometry}
\usepackage[bookmarks, hidelinks, unicode]{hyperref}
\usepackage[]{amsmath,amssymb}
\usepackage{stmaryrd}
\usepackage{tikz}
\usepackage{lmodern}
\usepackage{soul}
\usepackage{float}

% Packages configuration
\usetikzlibrary{shapes.arrows, angles, quotes}
\renewcommand{\arraystretch}{1.4}
\restylefloat{table}

% Shortcut commands
\newcommand{\im}{\text{Im}\,}
\newcommand{\re}{\text{Re}\,}
\newcommand{\img}{\text{Img}\,}
\newcommand{\R}{{\mathbb R}}
\renewcommand{\C}{{\mathbb C}}
\newcommand{\N}{{\mathbb N}}
\newcommand{\Z}{{\mathbb Z}}
\newcommand{\Q}{{\mathbb Q}}
\renewcommand{\U}{{\mathbb U}}
\newcommand{\cotan}{\operatorname{cotan}}
\newcommand{\conj}[1]{\overline{#1}}
\newcommand{\Aff}{\text{Aff}}
\newcommand{\twoRows}[1]{\multirow{2}{*}{#1}}
\newcommand{\threeRows}[1]{\multirow{3}{*}{#1}}
\newcommand{\twoCols}[1]{\multicolumn{2}{c|}{#1}}
\newcommand{\threeCols}[1]{\multicolumn{3}{|c|}{#1}}
\newcommand{\twoColsNB}[1]{\multicolumn{2}{c}{#1}}
\newcommand{\goesto}[2]{\xrightarrow[#1\:\to\:#2]{}}
\newcommand{\liminfty}{\lim_{x\to+\infty}}
\newcommand{\limminfty}{\lim_{x\to-\infty}}
\newcommand{\limzero}{\lim_{x\to0}}
\newcommand{ \const}{\text{cste}}
\newcommand{\et}{\:\text{et}\:}
\newcommand{\ou}{\:\text{ou}\:}
\newcommand{\placeholder}{\diamond}
\newcommand{\mediateur}{\:\text{med}\:}
\newcommand{\milieu}{\:\text{mil}\:}
\newcommand{\vect}[1]{\overrightarrow{#1}}
\newcommand{\point}[2]{(#1;\;#2)}
\newcommand{\spacepoint}[3]{\begin{pmatrix}#1\\#2\\#3\end{pmatrix}}
\newcommand{\sh}{\operatorname{sh}}
\newcommand{\ch}{\operatorname{ch}}
\renewcommand{\th}{\operatorname{th}}
\newcommand{\id}{\operatorname{id}}
\renewcommand{\cong}{\equiv}

% Document
\begin{document}

\begin{titlepage}
	\begin{center}
		\textit{\today}
		\vfill
		\textbf{\LARGE{Condensé de la MPSI}\\\Large{Mathématiques}}\\
		\vfill
		\large{Ewen Le Bihan\\MPSI -- Daudet}
	\end{center}
\end{titlepage}

\newpage
\tableofcontents

\newpage
\section{Processus de démonstration}
\subsection{Processus élémentaires}
\subsubsection{Quantification universelle $\forall$}
\label{quantification_universelle}
Soit $a \in E$

\subsubsection{Quantification existentielle $\exists$}
\label{quantification_existentielle}
Posons $a = \ldots \in E$

\subsubsection{Quantification existentielle unique $\exists!$}
\paragraph{Existence}
\emph{cf. \ref{quantification_existentielle}} 

\paragraph{Unicité}
Posons $b \in E$.
\emph{Démonstration de $b = a$} 

\subsubsection{Implication $P \implies Q$}
\label{implication}
Supposons $P(a)$. Montrons $Q(a)$

\subsubsection{Équivalence $P \iff Q$}
Procédons par double implication.

$\implies$: \emph{Démonstration de $P \implies Q$} 

$\impliedby$: \emph{Démonstration de $P \impliedby Q$} 

\subsubsection{Inclusion $E \subset F$}
\emph{Démontrer $\forall x \in \mathbb{E}, x \in E \implies x \in F$}.

\subsubsection{Égalité ensembliste}
Procédons par double inclusion.

$\subset$: Démonstration de $E \subset F$

$\supset$: Démonstration de $E \supset F$

\subsubsection{Égalité entre applications}
\emph{Démontrer $\forall x \in E,\ f(x) = g(x)$} 

\subsection{Processus de démonstration}
On commence chaque démonstration utilisant un de ces processus par << Procédons par \emph{nom du processus}  >> 

\subsubsection{Récurrence}
\emph{Pour montrer une propriété vraie dans $E \subseteq \N$} 

\paragraph{Initialisation}
\emph{Démontrer la propriété au premier rang} 

\paragraph{Hérédité}
Démontrer $\forall n \in E, P(n) \implies P(n+1)$

\paragraph{Conclusion}
La propriété étant initialisée et héréditaire, elle est vraie pour tout $n \in E$.

\subsubsection{Contraposée}
\emph{Pour montrer $P \implies Q$ quand l'implication directe est trop compliquée}

\emph{Démontrer $\lnot Q \implies \lnot P$} 

\subsubsection{l'Absurde}

\emph{Pour montrer $P$} 

Supposons $\lnot P$

$\vdots$

On obtient une contradiction.

On a donc $P$

\subsubsection{Disjonction des cas}

\paragraph{1er cas: \ldots} \ldots

\paragraph{2ème cas: \ldots} \ldots

$\vdots$

\paragraph{$n$-ième cas: \ldots} \ldots

\paragraph{Conclusion}
\ldots

\subsubsection{Analyse-Synthèse}
\emph{Pour trouver les solutions d'une équation, inéquation, \ldots}

\paragraph{Analyse}
Soit $a \in E$. Supposons $P(a)$.

\emph{Réduire le nombre de candidats possibles pour $a$} 

\paragraph{Synthèse}
Testons nos candidats

\paragraph{Conclusion}
Les solutions sont \ldots

\newpage
\section{Dérivation}
\emph{Attention aux hypothèses!} 
\subsection{Nombre dérivé en un point}
\[
	f'(a) = \lim_{x \to a} \frac{f(x)-f(a)}{x-a}
\] 

\subsection{Dérivée de $f$}
\[
	f' = \begin{cases}
		I\to \R \\
		a \mapsto f'(a)
	\end{cases}
\] 

\subsection{Dérivée usuelles}
\begin{itemize}
	\item $\forall n \in \N,\quad  (\text{id}^n)' = n \text{id}^{n-1}$
	\item $\forall n \in \N,\quad \sqrt[n]{\;}' = \frac{1}{n\sqrt[n]{\;} }  $ 
	\item $\ln' = \frac{1}{\text{id}}$
	\item $\exp'=\exp$
	\item $(a^\text{id})' = x\mapsto \ln(a)a^x$
	\item $ \sin' = \cos $ 
	\item $\cos' = -\sin$
	\item $\tan' = \frac{1}{\cos^2} = 1 + \tan^2$
	\item $ \sh' = \ch $
	\item $ \ch' = \sh$
	\item $\th' = \frac{1}{\ch^2} = 1 + \th^2$
	\item $\text{acos}' = \frac{-1}{\sqrt{1-\id^2} }$
	\item $\text{asin}' = \frac{1}{\sqrt{1-\id^2} }$
	\item $\text{atan}' = \frac{1}{1 + \id^2}$
\end{itemize}

\subsection{Dérivées de composées}
\begin{itemize}
	\item $\forall (\lambda, \mu) \in \R^2,\quad (\lambda u + \mu v)' = \lambda u' + \mu v'$
	\item $(uv)' = u'v+v'u$
	\item $(\frac{1}{v})' = \frac{-v'}{v^2}$
	\item $(\frac{u}{v})' = \frac{u'v-v'u}{v^2}$
	\item $(u \circ v)' = v' \cdot  (u' \circ v)$
	\item $(u^{-1})' = \frac{1}{u' \circ u^{-1}}$
\end{itemize}

\newpage
\section{Trigonométrie}
\subsection{Cercle trigonométrique ou unité $\mathcal C$}

Cercle de centre $\point{0}{0}$ et de rayon $1$.

\[
	\mathcal C = \{ \point{x}{y} \in \R^2,\: x^2 + y^2 = 1 \} = \{ \point{\cos x}{\sin x},\: x \in \R \} 
\] 

\subsection{Congruence $\cdot \cong \cdot [\cdot]$}

\[
	a \cong b\ [t] \stackrel{\text{def}}{\iff} \exists k\in \Z,\ a = b + kt

\] 
\subsubsection{Propriétés}

\begin{itemize}
	\item $\forall a, b, c, d\in \R,\ \begin{cases}
			a \cong b\ [t]\\
			c \cong d\ [t]
		\end{cases} \implies a + c \cong c + d\ [t]$
	\item $\forall a, b, \lambda \in \R,\ a\cong b\ [t] \implies \lambda a \cong \lambda b\ [\lambda t] \ \et \begin{cases}
			\lambda a \cong \lambda b\ [t] \\
			\lambda \in \Z
	\end{cases}$
\item $ \cdot \cong \cdot \ [ \cdot ]$ est une relation d'équivalence
\end{itemize}

\subsection{cos, sin, tan, cotan}
\begin{center}
\begin{tikzpicture}[scale=4]
	\draw (0, 0) circle (1);
	\draw (-1.25, 0) -- (1.25, 0);
	\draw (0, -1.25) -- (0, 1.25);
	\draw (1, -1.25) -- (1, 1.25);
	\draw (-1.25, 1) -- (1.732051, 1);
	\coordinate (O) at (0, 0);
	\draw (O) node[above, left, fill=white] {$O$};
	\coordinate (M) at (0.8660254, 0.5);
	\draw (M) node[above, fill=white] {$M$};
	\coordinate (I) at (1, 0);
	\coordinate (J) at (0, 1);
	\draw (I) node[right, fill=white]{$I$};
	\draw (J) node[above, fill=white]{$J$};
	\draw (O) -- (M);
	\draw (M) -- (1, 0.5773503) node[right, fill=white]{$\tan \theta$};
	\draw (M) -- (0.8660254, 0) node[below, fill=white]{$\cos \theta$};
	\draw (M) -- (0, 0.5) node[left, fill=white]{$\sin \theta$};
	\draw (M) -- (1.732051, 1) node[below]{$\cotan \theta$};
\end{tikzpicture}
\end{center}

\subsubsection{Théorème de Pythagore}
\[
	\cos^2 + \sin^2 = 1
\] 

\subsubsection{Théorème de Thalès}

\begin{align*}
	\tan = \frac{\sin}{\cos} \qquad \cotan = \frac{\cos}{\sin}
\end{align*}

\emph{Ce qui permet de trouver $\mathcal{D}_\tan$ et $\mathcal{D}_\cotan$} 

\subsubsection{Propriétés}

\begin{table}[h]
	\centering
	\begin{tabular}{lllll}
		 & périodicité & positif sur \footnote{ajouter $+\text{périodicité}$ aux bornes}& parité & domaine de définition \\\hline
		$\cos$ & $2\pi$ &  $[-\frac{\pi}{2} , \frac{\pi}{2} ]$ & paire & $\R$ \\
		$ \sin$ & $2\pi$ &  $[0 , \pi ]$ & impaire & $\R$ \\
		$\tan$ & $\pi$ &  $[0, \frac{\pi}{2}[$ & impaire & $\bigcup_{k \in \Z} ]-\frac{\pi}{2}+k\pi, \frac{\pi}{2}+k\pi[$ \\
		$\cotan$ & $\pi$  & $]0, \frac{\pi}{2}] \cup [-\frac{\pi}{2}, \pi[$ & impaire & $\bigcup_{k \in \Z} ]k\pi, \pi+k\pi[$
	
	\end{tabular}
	\caption{Propriétés des quatres fonctions trigonométriques}
	\label{tab:proprietes_quatres_fonctions_trigonométriques}
\end{table}

%TODO: valeurs remarquables
%TODO: formulaire

\subsubsection{Limite de $ \frac{\sin}{\id}$ en $0$}

\[
	\frac{\sin x}{x} \xrightarrow[{x\to 0}]{}  1
\] 
\subsection{acos, asin, atan}

\[
	\begin{cases}
		\forall x \in [-1, 1],\ &\exists! y \in [0, \pi],\ \cos y = x\\
		\forall x \in [-1, 1],\ &\exists! y\in [-\frac{\pi}{2}, \frac{\pi}{2}],\ \sin y = x \\
		\forall x \in \R,\ &\exists! y\in\ ]\!-\!\frac{\pi}{2}, \frac{\pi}{2}[,\ \tan y = x
	\end{cases}
\] 

\subsection{Équations trigonométriques}
\[
	\begin{cases}
		\cos x = a &\iff
			\begin{cases}
				a \in \{\operatorname{acos} a + 2k\pi,\ k \in \Z\}
					\cup \{\operatorname{acos} a + 2k\pi,\ k\in \Z\}
					& \text{si}\  a \in [-1, 1]\ \\
				\emptyset & \text{sinon}
			\end{cases}\\
		\sin x = a &\iff \begin{cases}
				a \in \{\operatorname{asin} a + 2k\pi,\ k\in \Z\}
				\cup \{\pi - \operatorname{asin} a + 2k\pi,\ k\in \Z\}
				& \text{si}\ a \in [-1, 1] \\
			\emptyset & \text{sinon}
		\end{cases} \\
		\tan x = a &\iff a \in \{\operatorname{atan} a + k\pi, k \in \Z\} 
	\end{cases}
\] 

\subsection{Amplitude $C$ \& déphasage $\phi$}

\[
	\forall A, B \in \R,\ \exists C, \phi \in \R,\ \forall x \in \R,\ A\cos x + B \sin x = C \cos(x-\phi) \\
\] 

\[
	
	\begin{cases}
		C > 0 \implies &\text{$C$ est l'amplitude} \\
					   &\text{$\phi$ est le déphasage}
	\end{cases}
\] 
\subsection{Identités remarquables}
\begin{itemize}
	\item $\forall x \in [-1, 1],\ \operatorname{acos} x + \operatorname{asin} x = \frac{\pi}{2}$
	\item $\forall x \in \R^\ast,\ \operatorname{atan} x + \operatorname{atan} \frac{1}{x} = \frac{\pi}{2}$
\end{itemize}


\newpage
\section{Logique}

\subsection{Table de vérité}

\begin{table}[H]
	\centering
	\begin{tabular}{ccc|c}
	Variable 1 & $\cdots$ & Variable $n$ & Formule \\\hline
	$v$ & $\cdots$ & $v$ & $\ldots$ \\
	$\vdots$ & ($2^n$ lignes) & & $\ldots$\\
	$f$ & $\cdots$ & $f$ & $\ldots$
	\end{tabular}
	\caption{Table de vérité pour une formule à $n$ variables}
	\label{tab:table_verité_cas_général}
\end{table}

\subsection{Connecteurs $\land$ $\lor$ $\lnot$, relations $\implies$ $\iff$}

\begin{table}[H]
	\centering
	\begin{tabular}{cc|cccc}
	$P$ & $Q$ & $P\land Q$ & $P \lor Q$ & $P \implies Q$ & $P \iff Q$ \\\hline
	$v$ & $v$ & $v$ & $v$ & $v$ & $v$ \\
	$v$ & $f$ & $f$ & $v$ & $f$ & $f$ \\
	$f$ & $v$ & $f$ & $v$ & $v$ & $f$ \\
	$f$ & $f$ & $f$ & $f$ & $v$ & $v$ \\
	\end{tabular}
	\caption{Table de vérité pour $\land$, $\lor$, $\implies$ et $\iff$}
	\label{tab:table_vérité_et_ou_implication_équivalence}
\end{table}

\begin{table}[H]
	\centering
	\begin{tabular}{c|c}
	$P$ & $\lnot P$ \\\hline
	$v$ & $f$ \\
	$f$ & $v$
	\end{tabular}
	\caption{Table de vérité pour $\lnot$}
	\label{tab:table_vérité_non}
\end{table}

\subsection{Égalité sémantique}

$(P = Q) \iff \text{$P$ a la même table de vérité que $Q$}$

\subsection{Propriétés des connecteurs $\land$ $\lor$ $\lnot$}

\newcommand{\landor}{\stackrel{\land}{\lor}}
\newcommand{\lorand}{\stackrel{\lor}{\land}}

\emph{Pour $\lor$ et $\land$} 
\begin{description}
	\item[Idempotence] $P \landor P = P$
	\item[Commutativité] $P \landor Q = Q \landor P$
	\item[Associativité] $P \landor (Q \landor R) = (P \landor Q) \landor R$
	\item[Distributivités] $P \lorand (Q \landor R) = (P \landor Q) \lorand (P \landor R)$
\end{description}

\emph{Pour $\lnot$}
\begin{description}
	\item[Involutivité] $\lnot \lnot P = P$
\end{description}

\subsection{Quantification existentielle unique $\exists!$}

\[
	[\exists! x \in E,\ P(x)] = [\underbrace{
			\exists x \in E,\ P(x)}_{\text{existence}} \land \underbrace{
			\forall \gamma_1, \gamma_2 \in E,\ P(\gamma_1) \land P(\gamma_2) \implies \gamma_1 = \gamma_2
	}_{\text{unicité}}]
\] 

\subsection{Négation $\lnot$}

\newcommand{\forallexists}{\stackrel{\forall}{\exists}}
\newcommand{\existsforall}{\stackrel{\exists}{\forall}}

\subsubsection{Négation de quantificateurs $\exists$, $\forall$}

\[
	\lnot(\forallexists x \in E,\ P(x)) = \existsforall x \in E,\ \lnot P(x)
\] 


\subsubsection{Négation de connecteurs ou lois de De Morgan}

\[
	\lnot(P \lorand Q) = \lnot P \landor \lnot Q
\] 

\subsubsection{Identités}
\begin{itemize}
	\item $P \land \lnot P = f$
	\item $P \lor \lnot P = v$
\end{itemize}

\subsection{Formules}

\begin{itemize}
	\item $P \implies Q = \lnot P \lor Q$
	\item $[\forall x \in \emptyset, P(x)] = v$
	\item $[\exists x \in \emptyset, P(x)] = f$
\end{itemize}

\newpage
\section{Équations différentielles}

\subsection{Recherche de la solution particulière $y_p$}


\begin{enumerate}
	\item Identifier la forme du second membre
	\item Exprimer $y_p$ avec des constantes inconnues
	\item Développer $y' + ay = \ldots$ avec $y = y_p$
	\item Trouver les constantes inconnues
	\item Exprimer $y_p$
\end{enumerate}

\subsubsection{Forme du second membre}

\begin{itemize}
	\item Combinaisaon linéaire $at+b$
	\item Constante $k$ \footnote{Ici l'expression de $y_p$ devient évidente: $y_p = t\mapsto \frac{k}{a}$}
	\item Polynôme du second degré $at^2+bt+c$
	\item Exponentielle $ke^{\gamma t}$
	\item "Trigonométrique" $\alpha \cos(\omega t) + \beta \sin(\omega t)$
\end{itemize}


\subsubsection{Second membre nul}

Second membre $= 0 \implies \begin{cases}
	\text{équation dite homogène} \\
	y_p = t\mapsto 0
\end{cases}$


\subsection{Premier ordre $y' + ay$}

\[
	\{t\mapsto ke^{-at} + y_p(t),\ k\in \R\} 
\] 

\subsection{Second ordre $ay'' + by' + cy$}

\paragraph{Équation caractéristique} $ar^2 + br + c$

\paragraph{Forme des solutions selon $\Delta$}

\begin{table}[h!]
	\centering
	\begin{tabular}{c|c|c}
	$\Delta > 0$ & $\Delta = 0$ & $\Delta < 0$ \\\hline
$Ae^{r_1t}+Be^{r_2t}$ & $(At+B)e^{r_0t}$ & $e^{\re(r_1)t}(A \cos(\im(r_1)t) + B \sin(\im(r_1)t))$
	\end{tabular}
	\caption{Forme des solutions d'une équadiff homogène du second ordre selon le signe de $\Delta$}
	\label{tab:solutions_equadiff_selon_signe_Delta}
\end{table}

\paragraph{Ensemble des solutions}

\[
	\{t\mapsto \text{forme des solutions},\ (A, B)\in \R^2\} 
\] 

\subsection{Problème de Cauchy}

\[
	\begin{cases}
		y' + ay &= k \\
		y'(b) &= c \\
	\end{cases}\quad\text{(premier ordre)}
	\qquad
	\begin{cases}
		ay'' + by' + cy &= k \\
		y''(\alpha) &= \beta \\
		y'(\gamma) &= \delta \\
	\end{cases}\quad\text{(second ordre)}
\] 

\begin{enumerate}
	\item Résoudre l'équadiff
	\item Résoudre l'équation ou le système en remplaçant $y$ par la forme des solutions
\end{enumerate}

\newpage
\section{Exponentielle imaginaire}

\subsection{Décomposition des fonctions à valeurs complexes $f = f_1+if_2$}
Soit $I \subseteq \R$
\[
	\forall f\in \C^I,\ \exists f_1, f_2 \in \R^I,\ f = f_1+if_2
\] 

\subsection{Relation fonctionnelle}

\begin{itemize}
	\item $\forall \theta_1, \theta_2 \in \R,\ e^{i(\theta_1+\theta_2)} = e^{i\theta_1} e^{i\theta_2}$
	\item $\forall \theta\in \R,\ e^{-i\theta} = \frac{1}{e^{i\theta}}$
\end{itemize}

\subsection{Euler}

\[
	\forall \theta\in \R,\ \cos\theta = \frac{e^{i\theta}+e^{-i\theta}}{2}\ \text{et}\ \sin\theta = \frac{e^{i\theta}-e^{-i\theta}}{2i}
\] 


\subsection{De Moivre}
\[
	\forall \theta\in \R,\ \forall n \in \Z,\ e^{in\theta} = (e^{i\theta})^n
\] 

\subsection{Linéarisation $\cos^n(\theta) = \sum \frac{?}{?} \cos(?\theta)$}

\emph{On cherche à linéariser $\cos^3$} 

\begin{enumerate}
	\item Euler \\
		$= \left( \frac{e^{i\theta}+e^{-i\theta}}{2} \right)^3$
	\item Binôme de newton \\
		$= 1\left( \frac{e^i{\theta}}{2} \right)^0\left( \frac{e^{-i\theta}}{2} \right)^3 + 3\left( \frac{e^{i\theta}}{2} \right)^1 \left( \frac{e^{-i\theta}}{2} \right)^2 + 3\left( \frac{e^i{\theta}}{2} \right)^2\left( \frac{e^{-i\theta}}{2} \right)^1 + 1\left( \frac{e^{i\theta}}{2} \right)^3 \left( \frac{e^{-i\theta}}{2} \right)^0$
	\item Moivre \\
		$= \dfrac{e^{-3i\theta}}{2^3} + 3\dfrac{e^{i\theta}}{2}\dfrac{e^{-2i\theta}}{2^2}+ 3\dfrac{e^{2i\theta}}{2^2}\dfrac{e^{-i\theta}}{2} + \dfrac{e^{3i\theta}}{2^3}$ \\
		$= \dfrac{1}{2^2}\dfrac{e^{3i\theta} + e^{-3i\theta}}{2} + \dfrac{3}{2^2}\dfrac{e^{i\theta} + e^{-i\theta}}{2}$
	\item Euler (réciproque) \\
		$= \frac{1}{4}\cos(3\theta) + \frac{3}{4}\cos(\theta)$
\end{enumerate}

\subsection{Arc-moitié $e^{i\ldots} + e^{i\ldots} = e^{\ldots}(e^{\ldots} + e^{\ldots})$}

\[
	\forall \theta_1, \theta_2 \in \R,\ e^{i\theta_1}+e^{i\theta_2} = e^{i\frac{\theta_1+\theta_2}{2}}(e^{i\frac{\theta_1-\theta_2}{2}}+e^{i\frac{\theta_2-\theta_1}{2}})
\] 

\subsection{Forme exponentielle $re^{i\theta}$}
\[
	\forall z\in \C^{\ast},\ z = |z|\exp(i\arg z)
\] 
\subsubsection{Égalité}
\[
	\forall z_1, z_2 \in \C_+^{\ast},\ z_1 = z_2 \iff \begin{cases}
		|z_1| &= |z_2| \\
		\arg z_1 &\cong \arg z_2\ [2\pi] \\
	\end{cases}
\] 

\subsection{Propriétés de $\arg$}

\emph{Identiques à celle de $\ln$, mais avec $[2\pi]$ et $\C^{\ast}$ à la place de $\R_+^\ast$} 

\subsection{Racines $n$-ième de l'unité $\U_n$}

\[
	\U_n &= \{\omega\in \C,\ \omega^{n} = 1\} = \{e^{\frac{2ik\pi}{n}},\ k\in \llbracket 0, n-1 \rrbracket \} 
\] 
\subsection{Résolution de $z^{n} = c$}

\[
	\forall c \in \C_+^{\ast},\ \forall n\in \N^{\ast},\ \{ z \in \C,\ z^n = c \} = \left\{ \sqrt[n]{|c|}\exp\left(i\frac{\arg c + 2k\pi}{n}\right),\ k \in \llbracket 0, n-1 \rrbracket \right\} 
\] 
\subsection{Dérivée d'une exponentielle complexe}

\[
	\forall \phi \in \C^{I},\ (\exp\circ\phi)' = \phi' \cdot \exp\circ\phi
\] 


\end{document}

