\documentclass{article}
\usepackage{amsmath, amssymb, mathrsfs, xfrac, txfonts, amsfonts}
\usepackage{tkz-tab}
\usepackage{graphicx ,caption}
\usepackage[a4paper, total={7in,10in}]{geometry}
\usepackage{lmodern}
\usepackage[bookmarks, hidelinks]{hyperref}
\usepackage{array}
\def\arraystretch{2}
\newcolumntype{C}{>{$}c<{$}}
\newcolumntype{L}{>{$}l<{$}}
\newcolumntype{R}{>{$}r<{$}}
\DeclareSymbolFont{matha}{OML}{txmi}{m}{it}% txfonts
\DeclareMathSymbol{v}{\mathord}{matha}{118}
\newcommand{\R}{\mathbb{R}}
\newcommand{\vect}[1]{\overrightarrow{#1}}
\newcommand{\rect}[1]{\square}
\newcommand{\veci}{\vec\imath}
\newcommand{\vecj}{\vec\jmath}
\begin{document}
\begin{titlepage}
\begin{center}
\textit{\today}
\vfill
\textbf{\LARGE{Condensé de la 1ère}\\\Large{Mathématiques}}\\
\vfill
\large{Ewen Le Bihan\\1eS3}
\end{center}
\end{titlepage}
\section*{Notations non vues en cours}
\begin{tabular}{c|l}
	$:=$ & Égal par définition\\
	$\mathbb{A} \cap \mathbb{B}$ & Appartient à la fois à $\mathbb{A}$ et à $\mathbb{B}$\\
	$\lceil x \rceil$ & Arrondir $x$ à l'entier supérieur. ($\lceil 5.1 \rceil = 6$)\\
	$1.5$ & Séparateur ,\\
	$x\cdot y$ & Multiplication $\times$\\
	$\vec{v} \:\bot\: \vec{u}$ & $\vec{v}$ et $\vec{u}$ orthogonaux
\end{tabular}
\pagestyle{empty}
\newpage
\tableofcontents
\pagestyle{empty}
\newpage


\pagestyle{plain}
\setcounter{page}{1}
\section{Polynômes du second degré $a x^2 + bx + c$}
\subsection{$\Delta$: Trouver les racines}
\begin{flalign*}
\Delta := b^2-4ac\\\\
\begin{cases}
	 \Delta > 0 & x_{1,2}=\frac{-b\pm \sqrt{\Delta}}{4a} \\
	 \Delta = 0 & x_0=\frac{b}{2a}\\
	 \Delta < 0 & \emptyset\\
\end{cases}
\end{flalign*}
\subsection{Étudier le signe}
Le signe du polynôme est celui de $a$, et, si $\Delta > 0$, est celui de $-a$ entre $x_1$ et $x_2$
\subsection{$\alpha$,$\beta$: Trouver l'extremum}
\subsubsection{Maximum ou minimum ?}
\begin{center}
\begin{tabular}{c|c}
	$a > 0$ & minimum\\
	$a < 0$ & maximum\\	
\end{tabular}
\end{center}
\subsubsection{Calcul}
\begin{flalign*}
\alpha &:= \frac{b}{2a}\\
\beta &:= \frac{\Delta}{4a}\\
\text{Sommet} &=(\alpha ; \beta)
\end{flalign*}
\begin{center}
Le polynôme atteint un extremum en $\alpha$ de valeur $\beta$
\end{center}
\subsection{Différentes formes}
\begin{center}
\begin{tabular}{c|l}
	Canonique & $(x-\alpha)^2+\beta$\\
	\hline
	Factorisée & $a(x-x_1)(x+x_2)$ \\ & $a(x_0-x)^2$\\
	\hline
	Développée & $a x^2 + bx +c$
\end{tabular}
\end{center}
\newpage


\section{Vecteurs $\vec{v}$, équations cartésiennes $ax+by+c=0$}
\subsection{Colinéarité} 
\begin{flalign*}
\vec{v} \:\&\: \vec{u} \text{ colinéaires} &\iff x_uy_v-y_ux_v=0 \\
&\iff (u) \parallel (v)\\
&\iff \vec{u} = \lambda \vec{v}\;\;\; (\forall \: \lambda \in \mathbb{R})
\end{flalign*}
\subsection{Vecteur directeur}
\subsubsection{Équation réduite $y = mx+p$}
\begin{flalign*}
\vec{v}\binom{1}{m}
\end{flalign*}
\subsubsection{Équation cartésienne $ax+by+c = 0$}
\begin{description}
	\item[Vecteur directeur] $(-b; a)$
	\item[Coefficient directeur] $m = -\frac{a}{b}$
	\item[Vecteur normal] $(a; b)$
\end{description}
\subsection{Décomposer un vecteur}
\begin{flalign*}
(\forall \: \lambda, \lambda' \in \mathbb{R}) \;\;\; \vec{w} = \lambda \vec{v} + \lambda' \vec{u}\\
\end{flalign*}
% TODO finish this section
\subsection{Relation de Chasles}
\begin{flalign*}
\overrightarrow{AC} = \overrightarrow{AB} + \overrightarrow{BC}
\end{flalign*}

\newpage\section{Statistiques}
\subsection{Caractéristiques}
\begin{center}
\begin{tabular}{c|c|rl}
	Nom & Type & Définition \\
	\hline
	Effectif total & /& $N:=$ & $\sum_{i=0}^{p}n_i$\\
	Moyenne & Centrale & $\bar{x}:=$ &  $ \frac{1}{N}\sum_{i=0}^{p}n_ix_i$ \\
	Médiane & Centrale & $Me:=$ & 
$\begin{cases}$
$N \text{ pair}\;\;\; &\frac{1}{2}\left( x_{\frac{N}{2}} + x_{\frac{N+1}{2}}\right)\\$
$N \text{ impair}\;\;\;&x_{\left\lceil\frac{N}{2}\right\rceil}$
$\end{cases}$\\
	Mode & Centrale & $Mo:=$ &Valeur ou classe qui a l'effectif le plus grand\\
	Premier Quartil & Non-centrale & $Q_1:=$ & $ x_{\left\lceil\frac{N}{4}\right\rceil}$\\
	Troisième Quartil & Non-centrale & $Q_3:=$ & $ x_{\left\lceil\frac{3}{4}N\right\rceil}$\\
	Étendue & Dispersion & $e:=$ & $ x_{max} - x_{min}$\\
	Écart inter-quartil & Dispersion & & $Q_3 - Q_1$\\
	Variance & Dispersion & $V :=$ & $\frac{1}{N}\sum_{i=0}^{p}(n_ix_i^2) - \bar{x}$\\
	Écart type & Dispersion & $\sigma :=$ & $\sqrt{V}$
\end{tabular}
\end{center}
\subsection{Transformation de valeurs selon $y=mx+p$}
\begin{flalign*}
\bar{y} &= m\bar{x}+p\\
V_y&=m^2V_x\\
\sigma_y &= |m|\sigma_x
\end{flalign*}

\subsubsection{Démonstrations}
\begin{align*}
	\overline{y}&= \frac{1}{N}\sum_{n=1}^{p} n_i mx_i+p \\
				&= m\frac{1}{N}\sum_{n=1}^{p} (n_ix_i) + p \\
				&= m \overline{x}+p \\
\end{align*}

% TODO: finir ça ↓
% \begin{align*}
% 	V_y &= \frac{1}{N}\sum_{n=1}^{p} (n_i(mx_i+p)^2) - \overline{y}\\
% 	&=  \\
% \end{align*}

\begin{align*}
	\sigma_y &= \sqrt{V_y}  \\
		 &= \sqrt{m^2V_x}  \\
		 &= \sqrt{m^2} \sqrt{V_y}  \\
		 &= |m|\sigma_x \\
\end{align*}


\newpage\section{Probabilités}
\subsection{Notions}
\begin{center}
\begin{tabular}{c|c|l}
Nom & Symbole & Description\\\hline
Univers & $\Omega$ & Ensemble des issues possibles\\
Variable aléatoire & $X$ & Fonction qui renvoie un nombre aléatoire dans $\Omega$\\
\end{tabular}
\end{center}
\subsection{Loi de probabilité de $X$}
Exemple: 
\begin{itemize}
	\item $\Omega = \{0;1;2\}$
	\item $p(X=0)=p(X=2)=\frac{1}{4}$
	\item $p(X=1)=\frac{1}{2}$
\end{itemize}
\begin{center}
\begin{tabular}{r||c|c|c}
k & \;\;0\;\; & \;\;1\;\; & \;\;2\;\;\\
\hline
$p(X=k)$ & $\frac{1}{4}$ & $\frac{1}{2}$ & $\frac{1}{4}$
\end{tabular}
\end{center}
\subsection{Caractéristiques}
\begin{center}
\begin{tabular}{c|l|rl}
	Nom & Description & Formule\\
	\hline
	Espérance & Résultat moyen espéré & $E(X) :=$ & $\sum_{i=1}^{n}p_ix_i$\\
	Variance & & $V(X) :=$ & $\sum_{i=1}^{n}(p_ix_i) - E(X)^2$\\
	Écart type && $\sigma(X) :=$ & $\sqrt{V(X)}$
\end{tabular}
\end{center}
\subsection{Issues, évennements}
Exemple:\\
\begin{tabular}{r||c|c}
	$x_i$ & A & B\\
	\hline
	$p(X=x_i)$ & $\frac{1}{3}$ & $\frac{2}{3}$\\
\end{tabular}
\\\\\\
Calcul de l'issue $AB$ ($A \to B$):
\begin{flalign*}
p(AB) &= p(A)\cdot p(B)
\end{flalign*}
Calcul de l'évenement $\Theta$ << au moins une fois $A$ >>:
\begin{flalign*}
p(\Theta) &= p(AB) + p(BA) + p(AA)
\end{flalign*}
Calcul de l'évennement contraire $\bar{\Theta}$:
\begin{flalign*}
p(\bar{\Theta}) = 1 - p(\Theta)
\end{flalign*}


\newpage
\subsection{Loi binomiale $\mathscr{B}$}
\subsubsection{Définitions}
\begin{center}
\begin{tabular}{r|l}
	Épreuve de Bernoulli \\
	\hline
	Évenement << succès >> & $S$\\
	Évenement << échec >> & $\bar{S}$\\
	Probabilité de succès & $p := p(S)$\\
	Probabilité d'échec & $q := p(\bar{S})$\\&\;\;\;\;$= 1 - p$\\\\
	Schéma de Bernouilli & $\mathscr{B}(n;p)$\\\hline
	Nombre de répétitions & $n$\\
	Nombre de succès & $k$\\
	Univers & $\Omega = [0;n]\cap \mathbb{N}$
	
\end{tabular}
\end{center}
\subsubsection{Loi de $X$}
Si $X \sim \mathscr{B}(n;p)$
\begin{center}
\begin{tabular}{c||c|c|c|c|c}
	$k$ & 0 & 2 & 3 & ... & $n$\\\hline
	$p(X=k)$ & $\binom{n}{k}p^kq^{n-k}$ & $\binom{n}{k}p^kq^{n-k}$ & $\binom{n}{k}p^kq^{n-k}$
 & \ldots & $\binom{n}{k}p^kq^{n-k}$


\end{tabular}
\end{center}
\subsubsection{Caractéristiques}
$$\forall k \in [0;n] \cap \mathbb{N}$$
\begin{flalign*}
E(X)&=np\\
V(X)&=npq
\end{flalign*}
\newpage

\section{Suites $U_n$}
\subsection{Types de suites}
\subsubsection{Fonctionnelle}
$$U_n=2n$$
\subsubsection{Récursive}
\begin{flalign*}
	\begin{cases}
    	U_{n+1}&=U_0+U_n\\
    	U_0&=5
	\end{cases}
\end{flalign*}
\subsection{Suites remarquables}
\subsubsection{Arithmétiques}
$$
\begin{cases}
U_{n+1}&=U_n + r\\
U_0&=k\\
\end{cases}
$$$$
U_n=U_0+r\cdot n
$$
\subsubsection{Géométriques}
$$
\begin{cases}
U_{n+1}&=U_n\cdot q\\
U_0 &= k
\end{cases}
$$$$
U_n=U_0\cdot q^n
$$
\subsection{Sommes}
\subsubsection{Suites arithmétiques}
\begin{flalign*}
	\sum_{\mu=i}^{j}U_\mu=\frac{U_i+U_j}{2} \cdot (j-i+1)
\end{flalign*}
\subsubsection{Suites géométriques}
\begin{flalign*}
	\sum_{\mu=i}^{j}U_\mu=U_i\cdot \frac{1-q^{j-i+1}}{1-q}
\end{flalign*}
\subsection{Variations}
\subsubsection{Fonction associée}
\begin{flalign*}
(\forall n \in \mathbb{N}) & f:n\mapsto U_n
\\
\text{Si } f \nearrow/\searrow &\implies U_n \nearrow/\searrow
\end{flalign*}
\subsubsection{Méthode 2}
$$U_{n+1}-U_n\lessgtr0\iff U_n \searrow / \nearrow$$
\subsubsection{Méthode 3}
$$\frac{U_{n+1}}{U_n}\lesseqgtr 1 \iff U_n \nearrow / \searrow$$

\newpage
\section{Produit Scalaire $\vec{u} \cdot \vec{v}$}
Note: pour éviter les confusions, la multiplication normale est notée $\times$ dans ce chapitre.
\subsection{Calcul}
\begin{center}
$\vec{\mu}, \vec{\kappa}$  projetés orthogonaux de $\vec{u}$ et $\vec{v}$
\end{center}
\begin{flalign*}
    	\vec{u} \cdot \vec{v} &= ||\vec{u}|| \times ||\vec{v}|| \times \cos{(\vec{u};\vec{v})}\\
    	&= x_u\times x_v+y_u\times y_{v}\;\text{*}\\
		&= \vec{u} \:\bot\: \vec{v}\\
		&= \vec{\mu} \cdot \vec{\kappa}\\
		&= \frac{1}{2} \left(||\vec{u}||^2 + ||\vec{v}||^2 - ||\vec{v}-\vec{u}||^2\right) \text{**}
\end{flalign*}
*  Seulement dans un repère orthonormé\\
** Si $\vec{u}$ et $\vec{v}$ sont vecteurs directeurs des segments formant un triangle:
\begin{figure}[htp]
\centering
\includegraphics[scale=0.25]{prodscal_triangle_1}
\caption{}
\label{}
\end{figure}

\subsection{Multiplication de segments liés}
Si $A, H, B$ alignés dans cet ordre
\begin{flalign*}
\overrightarrow{AB} \cdot \overrightarrow{AC} = AB \times AH
\end{flalign*}
Sinon, si $H, A, B$ alignés dans cet ordre
\begin{flalign*}
\overrightarrow{AB} \cdot \overrightarrow{AC} = -AB \times AH
\end{flalign*}
\subsection{Identités remarquables}
\begin{flalign*}
||\vec{u} \pm \vec{v}||^2 &= ||\vec{u}||^2 \pm 2 \vec{v}  \cdot \vec{u} + ||\vec{v}||^2\\
(\vec{u}-\vec{v})(\vec{u}+\vec{v}) &=||\vec{u}||^2 - ||\vec{v}||^2
\end{flalign*}
\subsection{Angle aigu et obtu}
\setlength{\belowdisplayskip}{0pt}
$$\vec{u} \cdot \vec{v} > 0 \iff (\vec{u}; \vec{v}) \text{ aigu}\\$$
\begin{center}
	Et inversement
\end{center}

\newpage
\section{Étude de fonctions}
\subsection{Fonctions de bases}
\begin{center}
\begin{tabular}{C|CCC}
	x & -\infty & 0 & +\infty \\\hline
	x & \nearrow & 0 & \nearrow\\
	x^2 & \searrow & 0 & \nearrow\\
	\sfrac{1}{x} & \searrow & || & \searrow\\
	\sqrt{x} & ///// & 0 & \nearrow\\
	|x| & \searrow & 0 & \nearrow
\end{tabular}
\end{center}

\subsubsection{Démonstration: $\sqrt{}$ est croissante sur sont intervalle de définition}

Soit $a, b \in \mathbb{R}_+$ tels que $a < b$.

\begin{align*}
	\sqrt{a} - \sqrt{b} &= \frac{(\sqrt{a} - \sqrt{b} ) \cdot (\sqrt{a} +\sqrt{b} )}{\sqrt{a} +\sqrt{b} } \\
	&= \frac{a-b}{\sqrt{a} +\sqrt{b} } \\
\end{align*}

Or $ \sqrt{a} $ et $ \sqrt{b} $ sont positifs ou nuls, donc, comme $a < b \implies a \neq b$, $ \sqrt{a} +\sqrt{b} $ est strictement positif.

De plus, $a < b \iff a - b < 0$, donc $\frac{a-b}{\sqrt{a} +\sqrt{b} }$.

Finalement:

\begin{align*}
	\sqrt{a} - \sqrt{b} &< 0 \iff \sqrt{a} < \sqrt{b} \\
\end{align*}

L'ordre est conservé, $ \sqrt{}  $ est croissante sur $\mathbb{R}_+$

\subsection{Opérations sur fonctions}
$\leftrightarrows$ : Changement de variation\\
$\rightrightarrows$ : Même variation
\begin{center}
$\forall \: k \in \mathbb{R}$,\;
$\forall \: \lambda_+ \in \mathbb{R}^+$,\;
$\forall \: \lambda_- \in \mathbb{R}^- $\\

\begin{tabular}{C|CCC}
	u & -\infty & 0 & +\infty\\\hline
	u + k & \rightrightarrows & k & \rightrightarrows\\
	u \cdot \lambda_+ & \rightrightarrows & 0 & \rightrightarrows\\
	u \cdot \lambda_- & \leftrightarrows & 0 & \leftrightarrows\\
	\sqrt{u} & ///// & 0 & \nearrow\\
	\sfrac{1}{u} & \searrow & || & \searrow\\
\end{tabular}
\end{center}

\newpage
\subsection{Dérivées}
\subsubsection{Nombre dérivé $f'(a)$}
\begin{flalign*}
f'(a) = \lim_{h\to0} \frac{f(a+h) - f(a)}{h}
\end{flalign*}
\subsubsection{Tangeante $T$ au point $a$}
\begin{flalign*}
T:y=\underbrace{f'(a)}_{\text{coef dir}}(x-a)+f(a)
\end{flalign*}
\subsubsection{Dérivées remarquables}
\begin{center}

$\forall \: n \in \mathbb{N} \;\;\; $
\begin{tabular}{C|C}
	f(x) & f'(x)\\\hline
	\text{constante} & 0\\
	x^n & nx^{n-1}\\
	\sqrt{x} & 1/2 \sqrt{x}\\
	\sfrac{1}{x} & - 1/x^2\\
\end{tabular}
\end{center}
\subsubsection{Opérations sur les dérivées}
\begin{center}
$(u+v)'$ et $(ku)'$ fonctionne normalement.
\begin{flalign*}
(uv)' &= u'v+v'u\\
\left(\frac{u}{v}\right)' &= -\:\frac{v'u-u'v}{v^2}\\
%\left(\frac{1}{v}\right)' & -\frac{v'}{v^2}\\
\end{flalign*}
\end{center}
\subsubsection{Utilisations de $f'(x)$}
Sens de variation de $f$\\
\begin{center}
%TODO truc sur les tableaux de signes. Genre mx+p: -oo: sgn(-a), +oo: sgn(a)
Si $f$ dérivable sur $[I;J]$\\
\begin{tabular}{C|CC}
	x & I  & J \\\hline
	f'(x) & + & -\\\hline
	f(x) & \nearrow & \searrow
\end{tabular}
\end{center}
Extrema (\textit{et pas <<extremums>> bordel de merde})\\
\begin{flalign*}
\text{Trouver le(s) $x$ pour } f'(x) = 0
\end{flalign*}

\subsection{Positions relatives}

Soit $f$ et $g$ deux fonctions avec pour coubres représentatives $C_f$ et $C_g$

\begin{enumerate}
	\item Etudier le signe de $f(x) - g(x)$ pour tout $x$ dans $D_f \cap D_g$
	\item $\displaystyle \begin{cases}
			f(x) - g(x) > 0 & C_f \;\text{est au dessus de } C_g \;\text{à }x \\
			f(x) - g(x) > 0 & C_f \;\text{est au dessous de } C_g \;\text{à }x \\
			f(x) - g(x) = 0 & C_f\;\text{et}\;C_g\;\text{sont confondues} \;\text{en }x \\
	\end{cases}$
\end{enumerate}

\subsubsection{Justifier les positions relatives des courbes représentatives des fonctions identité, carré et racine carrée}

Soit, pour toute fonction $f$, $C_f$ sa courbe représentative.
Soit, pour $x$ dans $D_i = \R$, $i = x\mapsto x$, pour tout $x$ dans $D_c = \R$, $c = x\mapsto x^2$ et, pour tout $x$ dans $D_r = \R_+$, $r = x\mapsto \sqrt{x} $.\\

Soit $x \in C_i \cup C_c \cup C_r = \R_+$

Etudions d'abord la position relative de $C_i$ et $C_c$. On étudie le signe de $i(x) - c(x)$.

\begin{align*}
	i(x) - c(x) &= x - x^2 \\
				&= x(1-x) \\
\end{align*}

Or $x > 0$  et $1 - x$ est strictement positif pour $x > 1$. Pour $x = 1$, $x(1-x) = 0$ et, pour $x < 1$, $x(1-x) < 0$. On en conclu que, pour $x < 1$, $C_i$ est au dessus de $C_c$, puis que les courbes ont un point d'intersection à $x = 1$, puis que $C_i$ est en dessous de $C_c$ pour $x > 1$.\\

Etudions ensuite la position relative de $C_i$ et de $C_r$: on étudie le signe de $i(x) - r(x)$.

\begin{align*}
	i(x) - r(x)&= x - \sqrt{x}  \\
	&= x - x^{0.5} \\
	&= x(1 - x^{-0.5}) \\
	&= x\left(1 - \frac{1}{\sqrt{x} }\right) \\
\end{align*}

\begin{center}
\begin{tikzpicture}
	\tkzTabInit{$x$/1, $x$/1, $1-\frac{1}{\sqrt{x}}$/1, $i(x)-r(x)$/1}{0, 1, $+\infty$}
	\tkzTabLine{z, +, z, +, , }
	\tkzTabLine{d, -, z, +, , }
	\tkzTabLine{d, -, z, +, , }
\end{tikzpicture}
\end{center}

%TODO: finish

\newpage\section{Trigonométrie}
\subsection{Notions}
\begin{description}	
	\item[Radians] unité d'angle, 1 radian équivaut à 180 degrés
	\item[Mesure principale] mesure dans $[0; 2\pi[$ d'un angle. Par ex. la mesure principale de $4\pi$ est $0$.
	\item[Cercle trigonométrique] cercle de rayon 1 ($2\pi r \to 2\pi$)
	\item[Angle orienté] les angles ont un signe selon le sens avec lequel ils sont mesurés
\end{description}
\subsection{Valeurs remarquables}
\begin{figure}[htp]
\centering
\includegraphics[scale=0.25]{cercle-trigo-filled}
\caption{}
\label{}
\end{figure}

\subsection{Formules de trigonométrie}

%TODO demo: https://stileex.xyz/formules-trigonometrie/

\begin{align*}
	\cos(a \pm b) &= \cos a \cos b \mp \sin a \sin b \\
	\sin(a \pm b) &= \sin a \cos b \pm \cos a \sin b \\
	\tan(a \pm b) &= \frac{\tan a \pm \tan b}{1 \mp \tan a \tan b} \\
	\cos 2a &= \cos^2a-\sin^2a \\
			&= 2\cos^2a-1 \\
			&= 1-2\sin^2a \\
	\sin 2a &= 2\sin a \cos a \\
	\cos^2a&= \frac{1+\cos 2a}{2} \\
	\sin^2a&= \frac{1-\cos2a}{2} \\
\end{align*}

\subsubsection{Démonstration}
Soient $a, b \in \R$. Soit $(O, \vec i, \vec j)$ un repère orthonormé.

On note $A$ le point tel que $OA = 1$ et tel que l'angle $(\vec i, \vect{OA}) = a$. 

On note $B$ le point tel que $OB = 1$ et tel que l'angle $(\vec i, \vect{OB}) = a+b$. 

On note $A'$ le point tel que $OA' = 1$ et tel que l'angle $(\vec i, \vect{OA'}) = a + \frac{\pi}{2}$.

%TODO: Représentation graphique

On a donc
\begin{align*}
	\vect{OA} &= \cos a \veci + \sin a \vecj \\
	\vect{OB} &= \cos (a+b) \veci + \sin (a+b) \vecj \\
	\vect{OA'} &= \cos (a+\frac{\pi}{2}) \veci + \sin (a+\frac{\pi}{2}) \vecj \\
			   &= -\sin a \veci + \cos a \vecj
\end{align*}

Or, dans $(O, \vect{OA}, \vect{OA'})$:
\begin{align*}
	\vect{OB} &= \cos (b) \vect{OA} + \sin (b) \vect{OA'} \\
\end{align*}

Donc:

\begin{align*}
	\vect{OB} &= \cos a \cos b \veci + \sin a \cos b \vecj - \sin a \sin b \veci + \cos a \sin b \vecj \\
			  &= (\cos a \cos b - \sin a \sin b)\veci + (\sin a \cos b + \cos a \sin b)\vecj \\
\end{align*}

Par unicité des coordonnées dans $(O, \veci, \vecj)$, on a:

\begin{align*}
	\cos(a+b)&= \cos a \cos b - \sin a \sin b \\
	\sin(a+b)&= \sin a \cos b + \cos a \sin b \\
\end{align*}

\newpage\section{Géométrie}

\subsection{Médiane}
Médiane issue d'un sommet $A$ $\iff$ Droite passant par $A$ et par le milieu du côté opposé à $A$

\subsection{Centre de gravité d'un triangle}
Centre de gravité de $ABC$ $\iff$ Unique point $G$ tel que $\vect{GA}+\vect{GB}+\vect{GC}=\vec 0$

\subsection{Théorèmé de la médiane}
\begin{itemize}
	\item Les médianes d'un triangle se coupent au centre de gravité
	\item Soit $G$ le centre de gravité de $\mathrm \Delta ABC$ et $M \in {A, B, C}$. Pour chaque médiane du triangle issue du sommet $M$ $[M M']$, $\vect{M'G} = \frac{1}{3}\vect{M'M}$
\end{itemize}

\subsection{Équation de cercle}
Soit $(x_0,y_0)$ le centre du cercle et $r$ sont rayon.

\[
	r^2 = (x_0-x)^2 + (y_0-y)^2
\] 

Si l'on connait un point $A$ sur le cercle ainsi que son centre $O$, il faut trouver $r$:

\begin{align*}
	r &= \operatorname{distance}(A, O) \\
	  &= \sqrt{(A_x - O_x)^2+(A_y - O_y)^2}  \\
\end{align*}

 

\end{document}
