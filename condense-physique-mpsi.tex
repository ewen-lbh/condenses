% Basic stuff
\documentclass{article}
\usepackage[utf8]{inputenc}
\usepackage[a4paper, total={6.5in, 9.5in}]{geometry}
\usepackage[bookmarks, hidelinks, unicode]{hyperref}
\usepackage{amsmath, amssymb}
\usepackage{tikz}
% \usepackage{circuitikz}
\usepackage{lmodern}
\usepackage{nicefrac}
\usepackage{soul}

% Packages configuration
\renewcommand{\arraystretch}{1.4}

% Shortuct commands
\newcommand{\R}{\mathbb{R}}
\renewcommand{\d}{\operatorname{d}}
\renewcommand{\vec}{\overrightarrow}
\newcommand{\const}{\text{const}}

% Document
\begin{document}

\begin{titlepage}
	\begin{center}
		\textit{\today}
		\vfill
		\textbf{\LARGE{Condensé de la MPSI}\\\Large{Physique}}\\
		\vfill
		\large{Ewen Le Bihan\\MPSI -- Daudet}
	\end{center}
\end{titlepage}

\newpage
\tableofcontents
\newpage


\section{Approche cinématique du mouvement d'un point}
\subsection{Travail d'une force}
\[
	\delta W = \vec{\text{la force}} \cdot \d\vec{OM}
\] 

\[
	\d\vec{OM} = \vec v  \cdot \d t
\] 

\begin{align*}
	\delta W > 0 &\implies \text{travail moteur} \\
	\delta W < 0 &\implies \text{travail résistant}
\end{align*}

\[
	W_{AB}(\vec F) = \int_A^B \delta W \d t = \int_A^B \vec F \cdot \d\vec{OM}
\] 

\[
	\dot{\vec{F}} = \vec 0 \implies W_{AB}(\vec F) = \vec F \cdot \vec{AB}
\] 

\[
	W(\vec{R_n}) = 0
\]

\subsection{Puissance d'une force}

\[
	\mathcal{P}(\vec F) = \vec F  \cdot  \vec v 
\] 
\subsection{Énergie cinétique}
\[
	E_c(M/\mathcal{R}) = \frac{1}{2}mv(M/\mathcal{R})^2
\] 

\subsubsection{Théorème de l'énergie cinétique}

\begin{itemize}
	\item $\mathcal{R}\ \text{galiléen} \implies \dot E_c(M/\mathcal R)$
	\item $\underset{i\to f}{\Delta}E_c = \underset{i\to f}{W}(\vec F)$
\end{itemize}	

\subsubsection{Pendule}
\[
	\ddot\theta + \frac{g}{l} \sin\theta = 0
\] 

\subsection{Énergie potentielle}

\[
	\Delta E_p := -\underset{A\to B}{W}(\vec F)
\] 
\[
	E_{p_p} = mgz + \text{const}\quad\text{si $(Oz)$ vertical ascendant}
\] 
\[
	E_{p_e} = \frac{1}{2}k(x-l_0)^2+\text{const}
\] 

\subsubsection{Stabilité d'un équilibre}

$x_e := \operatorname{extremum}(E_p, x)$ (i.e. $x_e$ est une position d'équilibre)

\begin{align*}
	-(x-x_e)\frac{\d^2 {E_p}}{\d x^2}(x_e)\quad \begin{cases}
		> 0 &\implies \text{stable} \\
		< 0 &\implies \text{instable}
	\end{cases}
\end{align*}

\subsection{Force conservative}
\[
	\vec F\ \text{conservative} \iff \begin{cases}
		W(\vec F)\ &\text{indépendante du chemin suivi} \\
		\vec F &= F(x)\d x \\
		\exists E_p \ &\text{/}\ \begin{cases}
			W(\vec F) &= -\Delta E_p \\
			\delta W &= -\d E_p
		\end{cases} \\
	\end{cases}
\] 

Contraire: force dissipative ou non-conservative



\subsection{Énergie mécanique}

En référentiel Galiléen
\[
	\Delta E_m = W(\vec{F_{\text{dissipative}}})
\] 
\subsubsection{Interprétations}

\begin{align*}
	\Delta E_m &= 0 \\
	\iff E_m &= \text{const} \\
	\iff E_m\ &\text{se conserve} \\
	\iff M\ &\text{est conservatif}
\end{align*}

\begin{align*}
	\Delta E_m &= W(\vec{F_{\text{dissipative}}}) < 0 \\
	\implies &E_m\ \text{décroissante} \\
	\implies &\text{Il y a {\bf dissipation} de l'énergie}
\end{align*}

\newpage
\section{Dynamique}
\subsection{Force gravitationelle}

\[
	\vec{F_{M\to M'}} = -G \frac{mm'}{r^2}\vec u
\] 

\subsection{Force de rappel d'un ressort}
\[
	\vec F = -k(l(t)-l_0)\vec{u_{\text{etirement}}}
\] 

\subsection{Tension d'un fil $\vec T$}

Aucune formule caractéristique :/

\subsubsection{Pendule}

En repère cylindrique
\[
	\vec T = -T\vec{u_r}
\] 

\subsection{Frottements solides $\vec R$}

\begin{align*}
	\vec{{{R}_{{n}}}}&\text{/} \begin{cases}
		\operatorname{sens}{\vec{{{R}_{{n}}}}} &= -\text{sens}{\vec{P}} \\
		{\left|{\left|\vec{{{R}_{{n}}}}\right|}\right|} &= {\left|{\left|\vec{{P}}\right|}\right|}
	\end{cases}\\
		\vec{R_t} &\text{/} \begin{cases}
		\vec{R_t} &\perp \vec{R_n} \\
		\operatorname{sens}{\vec{R_t}} &= -\operatorname{sens}{\vec{v}} \\
	\end{cases} \\
	\vec{R} &:= \vec{R_t} + \vec{R_n}
\end{align*}


\begin{itemize}
	\item Sans frottements solides \\
		$\vec{R_t} = \vec{0} \iff \vec{R} = \vec{R_n}$
	\item Avec frottements solides \\
		$\vec{R_t} \neq \vec 0$
		\begin{itemize}
			\item Sans glissement \emph{(i.e. $M$ est immobile)}:
				\begin{align*}
				\vec{R_n} + \vec{R_t} + \vec{P} &= \vec 0 \\
				\nicefrac{R_t}{R_n} &< f
				\end{align*}
			\item Avec glissement
				\begin{align*}
				\vec{R_n} + \vec{R_t} + \vec{P} &\neq \vec 0 \\
				\nicefrac{R_t}{R_n} &> f
				\end{align*}
		\end{itemize}
\end{itemize}

\subsection{Frottements fluides}

\[
	\vec{f} = -\lambda \vec{v}
\] 

\[
	\vec{f} = -\alpha v^2 \frac{\vec{v}}{||\vec{v}||}
\] 
\subsection{Lois de Newton}
\begin{enumerate}
	\item Il existe des référentiels dits "Galiléens" dans lesquels une particule pseudo-isolée ou isolée a un MRU. Si la particule est initialement au repos, elle le reste.
	\item $\vec{F} = m\dot{\vec{p}} = m \frac{\d}{\d t}(m \vec{v})$ 
	\item $\vec{F}_{A\to B} = -\vec{F}_{A\to B}$
\end{enumerate}

\end{document}
